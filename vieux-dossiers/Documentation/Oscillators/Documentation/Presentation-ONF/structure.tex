%!TEX program = xelatex
\documentclass[10pt]{article}
\usepackage{fontspec}
\usepackage[top=1.25in, bottom=1.25in, left=1.25in, right=1.25in]{geometry}
\usepackage{graphicx}
\usepackage{polyglossia}
\usepackage{amsmath}
\setmainlanguage{french}
\newfontfamily{\lstsansserif}[Scale=1]{Inconsolata}
\usepackage[T1]{fontenc}
\usepackage{listings}
\DeclareTextCommandDefault{\nobreakspace}{\leavevmode\nobreak\ }
\makeatletter
\g@addto@macro\@floatboxreset{\centering}
\makeatother

\usepackage{color}
\definecolor{grey}{rgb}{0.95,0.95,0.95}
\definecolor{lightgray}{rgb}{.9,.9,.9}
\definecolor{darkgray}{rgb}{.4,.4,.4}
\definecolor{darkblue}{rgb}{0, 0.3, 0.82}
\definecolor{darkorange}{rgb}{0.8, 0.4, 0}
\definecolor{darkgreen}{rgb}{0, 0.55, 0.2}
\definecolor{darkpurple}{rgb}{0, 0.6, 0.6}
\definecolor{p5pink}{rgb}{0.85, 0, 0.5}
\setlength\marginparsep{10pt}
\setlength\marginparwidth{60pt}
\setlength{\parskip}{1em}

\lstdefinelanguage{JavaScript}{
    keywords={break, case, catch, continue, debugger, default, delete, do, else, false, finally, for, function, if, in, instanceof, new, null, return, switch, this, throw, true, try, typeof, var, void, while, with},
    keywordstyle=\color{darkblue}\lstsansserif,
    ndkeywords={class, export, boolean, throw, implements, import, this},
    ndkeywordstyle=\color{darkgray}\lstsansserif,
    identifierstyle=\color{black},
    sensitive=false,
    comment=[l]{//},
    morecomment=[s]{/*}{*/},
    commentstyle=\color{darkgray}\lstsansserif,
    stringstyle=\color{darkorange}\lstsansserif,
    morestring=[b]',
    morestring=[b]",
    keepspaces=true,
    showstringspaces=false,
    classoffset=1,
    morekeywords={background,blendMode,createVector,createCanvas,dist,ellipse,fill,
    lerp,line,map,rect,translate,
    abs,acos,asin,atan,atan2,cos,floor,round,sin,stroke,strokeWeight,tan,pow},
    keywordstyle=\color{p5pink}\lstsansserif
}

\lstset{backgroundcolor=\color{grey},
    numbers=left,
    columns=fullflexible,
    basicstyle=\footnotesize\lstsansserif,
    language=JavaScript,
    numberstyle=\footnotesize,
    numberstyle=\color{darkgray},
    xleftmargin=12pt,xrightmargin=12pt,
    aboveskip=12pt,belowskip=12pt,
    frame=tlbr,framesep=12pt,framerule=0pt,
    numbers=none,
    lineskip={1pt}
}

\lstset{literate=%
    *{0}{{{\color{darkgreen}0}}}1
    {1}{{{\color{darkgreen}1}}}1
    {2}{{{\color{darkgreen}2}}}1
    {3}{{{\color{darkgreen}3}}}1
    {4}{{{\color{darkgreen}4}}}1
    {5}{{{\color{darkgreen}5}}}1
    {6}{{{\color{darkgreen}6}}}1
    {7}{{{\color{darkgreen}7}}}1
    {8}{{{\color{darkgreen}8}}}1
    {9}{{{\color{darkgreen}9}}}1
}

\author{Guillaume Pelletier-Auger}
\title{La nuit géométrique - Synopsis}
\date{9 octobre 2016}

\begin{document}

\maketitle
%\newpage

% \begin{center}
% \end{center}
%!TEX root = structure.tex

% Un résumé du projet en huit lignes.
% Points importants du résumé : une description esthétique du film, une brève explication de sa fabrication (numérique, algorithme, logiciel libre). Une brève description de sa portée (code source libre, modifiable, création du film devant le public).

\begin{abstract}

Des milliers de points blancs dansent sur un fond noir, se groupant en formes harmonieuses puis se dégroupant en volées dissonantes. \textit{La nuit géométrique} est un film d'animation abstrait généré par une série d'algorithmes et programmé avec des logiciels libres. Le code source du film sera ouvert, documenté et distribué librement, permettant à quiconque d'en comprendre la génèse et d'en créer de nouvelles versions. Le film pourra être visionné et modifié même pendant sa conception. Cet hommage aux films abstraits de Mary Ellen Bute et de René Jodoin veut ainsi utiliser les nouvelles technologies afin d'inviter le public à découvrir le cinéma d'animation de manière participative et à réfléchir sur la liberté de l'information.

 % et les divers moyens par lesquelles celle-ci peut changer notre société.

% quand on dit que c'est important de prendre connaissance d'une chose... on fait de la .... conscientiser... awareness

% et se veut un plaidoyer pour l'accessibilité des moyens de production en animation


% Et de découvrir le cinéma abstrait de manière participative.


% \textit{La nuit géométrique} est film d'animation abstrait qui explore les mouvements d'un groupe de particules ressemblant à une galaxie. Des motifs se forment et se déforment dans une danse générée par une grande quantité d'algorithmes et de formules mathématiques. 
% Il s'agit d'une \oe{}uvre réalisée de façon algorithmique, c'est-à-dire que tous les mouvements des particules sont générés par une grande quantité d'algorithmes et de formules mathématiques déclenchées par du code informatique.

% La totalité du code source du film sera distribuée sur internet en code source ouvert ("open source"), ce qui permettra à quiconque de modifier le film, d'en créer une version entièrement différente, ou de créer une toute nouvelle \oe{}uvre qui s'en inspire. Il s'agit donc simultanément d'une exploration esthétique du mouvement et d'une réflexion sur l'acte créateur.

% , qui explore comment ce médium (le film abstrait) peut être ré-imaginé à une ère où il est possible de documenter la totalité de son travail, d'une façon telle que quiconque sur terre soit capable de savoir exactement comment le film est créé. Je veux démocratiser le film abstrait, le rend accessible et lui faire perdre son aura d'ésotérisme.

    % L'accessibilité est le mot clé.
\end{abstract}

% POUR LES CINÉASTES (PRODUCTIONS 100 % ONF)

% Un projet peut être soumis dès les premières étapes de développement : l’analyse préliminaire tiendra compte du degré d’avancement. Le dossier doit toutefois comprendre tous les éléments permettant d’apprécier le contenu et la production envisagée.

% Un résumé du projet en huit lignes.
% Un synopsis de quelques pages précisant :
%   le sujet et les enjeux;
%   l’approche cinématographique envisagée.
% Un rapport de recherche, si disponible.
% Les publics ciblés par la production.
% Un curriculum vitae, une biofilmographie et trois (3) copies DVD d’une œuvre antérieure.
% Les renseignements habituels sur le projet (durée, format, genre, etc.).
% Pour les projets de films d’animation, inclure, en plus des autres éléments, des éléments du scénarimage (storyboard) et la technique envisagée.
% Veuillez remplir le formulaire ci-joint lorsque vous déposez un projet au Programme français et l’attacher à votre documentation :

% \section{Présentation}
Ce film pose la question suivante : comment un film d'animation abstrait, comme ceux que réalisaient Norman McLaren et Mary Ellen Bute au milieu du vingtième siècle, pourrait être réalisé aujourd'hui? De quelle manière un tel film pourrait-il réfléter notre époque?

Nous vivons à une ère où les technologies deviennent de plus en plus accessibles, où les moyens de production cinématographique, par exemple, sont plus accessibles que jamais au grand public. Cependant, dans une très large mesure, ces moyens de productions sont encore réservés aux gens fortunés. 

% À une ère marquée par Internet et la démocratisation de l'information,
À l'ère de l'information et des communications internationales instantanées, il est intéressant d'imaginer un film abstrait comme étant une \oe{}uvre ouverte, dont la fabrication est entièrement documentée et dont le code source est disponible gratuitement à quiconque.

\newpage

% Structure :  Sujet et les enjeux
% L'information libre permet aux artistes de s'exprimer.
% Les nouveaux outils technologiques permettent l'accès aux moyens de production.
% 








\section{Le sujet et les enjeux}

% Le cinéma abstrait, même s'il ne présente aucune trame narrative ni aucun élément figuratif, est tout de même chargé de sens, un sens qui émerge des moyens et des circonstances de production du film ainsi que des préoccupations de ses créateurs. Par exemple, l'art brut des Polonais d'après-guerre leur permettrait de contester graphiquement la propagande soviétique.

\subsection{L'information libre}

% L'accès à l'information est un sujet qui remue beaucoup notre société. 

% M'inspirer de l'histoire de Processing et de la Processing Foundation.
% http://benfry.com/writing/archives/513
% https://processing.org/overview/

Si un code source bien documenté du film \textit{Notes sur un triangle} de René Jodoin existait, et s'il avait été produit avec des logiciels libres, n'importe qui aurait le loisir de décortiquer tous les moyens par lesquels ce film a été créé. Une petite fille curieuse de la Côte d'Ivoire pourrait prendre ce code et décider de réaliser sa propre version du film, et nous gagnerions à voir son \oe{}uvre autant que celle de Jodoin.

% Des privilèges rares tels que l'accès à un studio d'animation, ou même l'accès à de coûteux logiciels commerciaux, ne seraient plus des obstacles. 

% Nous vivons à une époque où de nombreux moyens d'éducation gratuite permettent à des gens de s'extirper de la pauvreté, de devenir des chercheurs scientifiques ou des artistes.

\textit{La nuit géométrique} est un projet rendu possible par des logiciels libres. En réalisant ce film, je veux expérimenter sur ce que peux être une \oe{}uvre libre, c'est-à-dire une \oe{}uvre sur laquelle je n'aurais aucun droit exclusif. Qu'arrive-t-il lorsqu'un artiste partage absolument toutes les facettes de sa création?

Je m'inspire de la fondation Processing, qui a 



% Ce film est tout d'abord un plaidoyer pour l'information libre. Il s'inspire du brillant programmeur Aaron Schwartz, auteur du \textit{Guerilla Open Access Manifesto} dans lequel il faisait un plaidoyer passionné et émouvant pour la libération de l'information et l'éducation de toutes et tous. ArXiv. Alexandra Elbakyan et son Sci-Hub, qui malgré qu'illégal, force la réflexion sur l'accès à l'information.


\subsection{Distribution du code source ouvert}
% Un important aspect de ce film.


Lorsque Butte, McLaren et les autres réalisaient leurs films, leurs créations étaient essentiellement des \textit{boîtes noires} : des \oe{}uvres fermées dont les méthodes de fabrication étaient cachées. Pour une immense majorité du public qui n'avait pas eu la chance de côtoyer ces créateurs dans leurs studios, et qui n'avaient pas non plus accès aux outils de production (caméras, bancs titres, oscilloscopes, etc.), ces \oe{}uvres-là, bien qu'inspirantes et belles, étaient aussi ésotériques et inaccessibles. Cryptiques. (Nul besoin de blâmer les artistes de cette époque, cependant, puisqu'ils travaillaient avec les moyens qui leur étaient accessibles.)

Maintenant, en réalisant une \oe{}uvre expérimentale en utilisant des logiciels libres et en publiant le code source gratuitement sur internet, l'\oe{}uvre peut \textit{devenir} autre chose. Elle est auto-documentée, il n'y a pas de grandes barrières entre son public et ses moyens de réalisation. Elle n'est pas cryptique.

De plus, le code source de ce film a été créé avec l'outil (libre) de contrôle des versions \textit{Git}, qui permet de conserver l'historique complète de toute la création du film. Chacune des centaines de formes que prendra le film à mesure que je le réalise sera conservée par ce système, et il sera possible à quiconque de consulter cette archive. Un cinéphile curieux pourrait regarder une version différente du film à chaque jour pendant que j'y travaille, puisque toutes les versions du film seront publiées en direct sur Internet pendant sa réalisation. Ce même cinéphile pourrait décider de prendre mes documents et d'entamer sa propre version du film, et la terminer avant même que je termine la mienne. Même le document que vous êtes en train de lire pourrait faire partie de cette documentation accessible à tous, des premières ébauches jusqu'au texte final.




\subsection{Une \oe{}uvre auto-documentée}
Une \oe{}uvre auto-documentée est une \oe{}uvre qui contient toute l'information nécessaire pour comprendre sa génèse intégralement. Le code source de ce film sera documenté pleinement, ce qui rendra son accès beaucoup plus aisé.

Je m'inspire ainsi du principe de \textit{programmation lettrée}, ou \textit{programmation littéraire} inventé par Donald Knuth en 1981. Le \textit{code} est ainsi d'abord écrit en langage naturel (en prose), compréhensible par tout le monde, avant d'être écrit sous la forme à laquelle nous associons normalement le code informatique.

Mary Elle Butte et Norman McLaren cherchaient déjà à rendre leur cinéma accessible en faisant une présentation textuelle claire et ouverte en préface à leurs films. Il s'agissait donc, déjà, d'une forme d'\oe{uvre auto-documentée. Et maintenant, avec les nouvelles technologies, cette auto-documentation peut devenir tellement plus importante qu'elle peut permettre à l'auditoire de déconstruire et reconstruire le film en entier chez eux. Il sagit donc, d'une certaine façon, de pousser plus loin des idées conçues par ces pionniers grâce aux nouveaux outils qui sont à notre portée. Si Norman McLaren avait pu partager chaque morceau de la mécanique de ses films et mettre ses outils dans les mains de son public, il l'aurait certainement fait.

% http://www.awn.com/mag/issue1.2/articles1.2/moritz1.2.html
% The kind of titles Mary Ellen used to preface her films, explaining them to an average audience as a new kind of art linking sight and sound prefigure McLaren's similar audience--friendly prefaces to his National Film Board experiments. Mary Ellen also proudly announced that she had used combs and collanders and whatever else to make the imagery in her films, encouraging a delight in simplicity and novelty of experimentation. Surely this left its mark on McLaren, too.





\section{Approche cinématographique}

% Structure : Approche cinématographique
% C'est un film abstrait créé avec des algorithmes et des formules mathématiques, avec JavaScript et la bibliothèque p5.js, deux outils gratuits.
% Le film est héritier des Norman McLaren, René Jodoin... Le film est une séquence très minimaliste de points blancs sur fond noir avec lesquels je crée de nombreuses scènes distinctes, qui auront un caractère distinct, et qui suivent la musique. Il est intéressant de voir, par exemple, comment Norman McClaren a réussi à créer une séquence filmique variée avec de simples lignes ans \textit{Lignes horizontales} et \textit{lignes verticales}. 

%La musique que j'imagine pour l'instant utiliser (mais dont je ne suis pas certain de pouvoir obtenir les droits) est une composition d'Oskar Sala...

% Mon programme génère lui-même les images qui serviront au montage final.
%
% Il va sans dire que je pourrais réaliser ce film à l'extérieur de l'ONF (logiciels gratuits), et je sais que l'ONF privilégie les films qui ne pourraient être réalisés qu'avec son aide. Cependant je crois qu'il pourrait être fascinant de partager cette aventure avec l'ONF et que l'ONF participe à conscientisation des outils de production filmique en code source libre. Que l'ONF fasse comprendre aux gens que le film d'animation peut se faire par toutes et tous.


\subsection{Conception et production}
Ce film sera réalisé en écrivant du code informatique qui génère ses images. Le film sera écrit spécifiquement en langage JavaScript et utilisera la bibliothèque p5.js. Cette bibliothèque est une série d'outil dédiée à la génération d'images. Le code lui-même consistera en une grande série de formules mathématiques et d'algorithmes qui seront enchaînés les uns après les autres. Chacun des algorithmes génèrent plusieurs milliers particules lumineuses qui sont animées sur un fond noir, comme une grande danse d'étoiles au creux d'une galaxie. 

Mon approche cinématographique s'inspire du cinéma abstrait des années 30 à 70 : le cinéma expérimental de Norman McLaren, de Mary Ellen Bute, de Jordan Belson, de John et James Whitney, etc. Larry Cuba, Oskar Fischinger, Harry Smith.

Le principe de \textit{musique visuelle} proné par Mary Ellen Bute m'est particulièrement inspirant. Le film fonctionne avec de la musique et les points ont l'air de danser une grande valse stellaire.

Mary Ellen Bute et Norman McLaren se servaient parfois d'images générées par un oscilloscope. J'utilise des procédés semblables, mais mon oscilloscope est recréé par programmation informatique.

\subsection {Musique et ambiance sonore}
La musique est une part fondamentale de ce film, comme elle l'est dans beaucoup de film abstrait.

\newpage
\section{Répercussions et suites}
Ce projet est très différent des \oe{}uvres de cinéma d'animation traditionnel que j'ai produites dans le passé et qui découlaient directement de mes études en animation au Cégep du Vieux Montréal.

Pour la suite, j'ai plusieurs idées pour entre-mêler ces inspirations drastiquement différentes. Mélanger l'animation abstraite et algorithmique avec des éléments davantage figuratifs et une trame narrative. Par exemple, je compte créer un engin qui me permettrait de réaliser des animations en papier découpé entièrement avec du logiciel libre et gratuit. Il s'agirait d'une extension à la bibliothèque p5.js, et en la créant pour réaliser un premier film de ce genre, je distribuerais bien entendu le code source pour permettre à quiconque de réaliser des films de ce genre sans dépenser le moindre sou pour des logiciels. À l'ère où nous vivons, réaliser un film d'animation ne devrait pas être plus coûteux qu'écrire un recueil de poésie. Lorsque les moyens de création deviennent plus accessibles, c'est toute une nouvelle population qui gagne l'opportunité de s'exprimer.



\newpage
\section{Les publics ciblés par la production}
%Accessible à toutes et à tous. Possiblement intéressant pour un jeune public qui pourrait s'intéresser à la programmation ou aux mathématiques et ne pas réaliser tout le potentiel et l'étendue de ces disciplines (Comme tout sera expliqué et vulgarisé avec la distribution libre du code source, etc.). Je voudrais mettre l'emphase sur le fait que ce film abstrait et géométrique, qui est un genre de cinéma parfois un peu ésotérique et loin du grand public, est accessible à toutes celles et à tous ceux qui s'y intéressent.


Mon objectif est de créer une \oe{}uvre qui soit accessible à toutes et à tous. Je crois que le public enfant et adolescent peut tirer autant de plaisir au cinéma abstrait que le public adulte. Également, je souhaite que ce film fasse savoir à son public que ses moyens de production lui sont accessibles. Un des buts initiaux de la bibliothèque p5.js et de la Processing Foundation qui gère son développement et de faire un outil qui puisse initier tout le monde à la programmation, incluant les enfants. Des cours de programmations offerts aux enfants utilisent régulièrement p5.js. Le film pourrait avoir une notice au début ou à la fin, du genre "Ce film est \oe{}uvre dont la source peut être lue, modifiée et redistribuée comme bon vous semble, gratuitement et en toute liberté."

Il pourrait s'agir d'une belle opportunité de promouvoir le film d'animation comme étant une chose qui peut être faite par le public, qui n'est pas réservée aux rares gens qui ont la chance d'avoir accès à des outils exclusifs et coûteux.




% %!TEX root = structure.tex
\chapter{L'algorithme de Prim}
\begin{figure}[h]
\includegraphics[width=1\textwidth]{images/algorithme_de_prim_001}
\caption{Les élégantes formes s'enchaînent.}
\end{figure}

Le premier algorithme que j'ai exploré est celui inventé par Prim. Il s'agit d'un \textit{arbre couvant de poids minimal}. C'est cet algorithme qui m'a donné envie d'en explorer mille autres. Cet algorithme trouve un sous-ensemble d'arêtes formant un arbre sur l'ensemble des sommets du graphe initial, et tel que la somme des poids de ces arêtes soit minimale. 

\newpage
\begin{lstlisting}
// Cet algorithme prend un array nommé "vertices" qui contient des vecteurs avec des valeurs x et y.

var unreached = vertices.slice();
var reached = [];
reached.push(unreached[0]);
unreached.splice(0, 1);

(function findEdges() {
var record = 100000;
var rIndex;
var uIndex;
for (var i = 0; i < reached.length; i++) {
  for (var j = 0; j < unreached.length; j++) {
    var v1 = reached[i];
    var v2 = unreached[j];
    var d = dist(v1.x, v1.y, v2.x, v2.y);
    if (d < record) {
      record = d;
      rIndex = i;
      uIndex = j;
    }
  }
}
drawEdge(reached[rIndex], unreached[uIndex]);

reached.push(unreached[uIndex]);
unreached.splice(uIndex, 1);
if (unreached.length > 0 && !stopped) {
  setTimeout(function() {
    findEdges();
  }, 1);
}
}());
\end{lstlisting}
L'\oe{}uf de Louvicourt
\begin{lstlisting}
for (var i = 0; i < 1000; i++) {
    t = i/10;
    x = cos(t) * sin(t/2.1) * 400 * i/1000;
    y = sin(t) * 350;
    x += width/2;
    y += height/2;
    var v = createVector(x, y);
    vertices.push(v);
}
\end{lstlisting}


\chapter{Près du Richelieu évasif}
Demain, dès l'aube, j'irai par les sentiers fouler l'herbe menue. Je laisserai le vent baigner ma tête nue. J'irai seul par la nature. Heureux comme avec une femme.
\section{Rêver est une chose importante}
De la plus haute importance, dirais-je même. Les étoiles sont là pour ça. Les mers sont bondées de beaux liquides qui nous calment, qui guérissent nos meurtrissures. Les mers sont nos baumes. Les mers sont nos beaux horizons aperçus à vélo. La beauté majestueuse de l'océan Atlantique qui se jette dans nos yeux et baignent nos nerfs oculaires meurtris. Il nous est permis de rêver.
\begin{lstlisting}
var p = [];

for (var i = 0; i < 3300; i++) {
    var string="Oh yeah! J'aime 30 soleils!";
    var t = i/50;
    var x = sin(t) * cos(t * 2) * 560 * (i / 1050);
    var y = pow(cos(t),11) * 760;
    var v = createVector(x, y);
    p.push(v);
}
\end{lstlisting}
\section{Rêver est une chose importante}
De la plus haute importance, dirais-je même. Les étoiles sont là pour ça. Les mers sont bondées de beaux liquides qui nous calment, qui guérissent nos meurtrissures. Les mers sont nos baumes. Les mers sont nos beaux horizons aperçus à vélo. La beauté majestueuse de l'océan Atlantique qui se jette dans nos yeux et baignent nos nerfs oculaires meurtris. Il nous est permis de rêver. J'implore Przemys\l aw Prusinkiewicz de m'aider dans ma quête majestueuse.
\begin{lstlisting}
var p = [];

for (var i = 0; i < 3300; i++) {
    var t = i/50;
    var x = sin(t) * cos(t * 2) * 560 * (i / 1050);
    var y = pow(cos(t),11) * 760;
    var v = createVector(x, y);
    p.push(v);
}
\end{lstlisting}
% \input{introduction}

\end{document}